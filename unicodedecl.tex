% Unicode-Zeichen direkt in Formeln benutzen (Falls sie auf dem jew. Keyboard vorhanden sind - falls nicht nimmt das hier nur Platz weg)

%\DeclareUnicodeCharacter{00A3}{\ensuremath{\bigtriangleup}}    %£
%\DeclareUnicodeCharacter{20AC}{\ensuremath{\bigtriangleup}}    %€
\DeclareUnicodeCharacter{0398}{\ensuremath{\Theta}}             %Θ
\DeclareUnicodeCharacter{03A0}{\ensuremath{\prod}}              %Π
\DeclareUnicodeCharacter{0393}{\ensuremath{\Gamma}}             %Γ
\DeclareUnicodeCharacter{039B}{\ensuremath{\Lambda}}            %Λ
\DeclareUnicodeCharacter{03A6}{\ensuremath{\Phi}}               %Φ
\DeclareUnicodeCharacter{1D60}{\ensuremath{\varphi}}            %ᵠ
\DeclareUnicodeCharacter{2207}{\ensuremath{\nabla}}             %∇
\DeclareUnicodeCharacter{2206}{\ensuremath{\bigtriangleup}}     %∆
\DeclareUnicodeCharacter{0394}{\ensuremath{\Delta}}             %Δ
\DeclareUnicodeCharacter{03A3}{\ensuremath{\sum}}               %Σ
\DeclareUnicodeCharacter{2202}{\ensuremath{\partial}}           %∂
\DeclareUnicodeCharacter{222B}{\ensuremath{\int}}               %∫
\DeclareUnicodeCharacter{221E}{\ensuremath{\infty}}             %∞
\DeclareUnicodeCharacter{00B1}{\ensuremath{\pm}}                %±
\DeclareUnicodeCharacter{03B1}{\ensuremath{\alpha}}             %α
\DeclareUnicodeCharacter{03B2}{\ensuremath{\beta}}              %β
\DeclareUnicodeCharacter{03B3}{\ensuremath{\gamma}}             %γ
\DeclareUnicodeCharacter{03B4}{\ensuremath{\delta}}             %δ
\DeclareUnicodeCharacter{03B5}{\ensuremath{\epsilon}}           %ε
\DeclareUnicodeCharacter{03B6}{\ensuremath{\zeta}}              %ζ
\DeclareUnicodeCharacter{03B7}{\ensuremath{\eta}}               %η
\DeclareUnicodeCharacter{03D1}{\ensuremath{\vartheta}}          %ϑ
\DeclareUnicodeCharacter{03BA}{\ensuremath{\kappa}}             %κ
\DeclareUnicodeCharacter{03BB}{\ensuremath{\lambda}}            %λ
\DeclareUnicodeCharacter{03BC}{\ensuremath{\mu}}                %μ
\DeclareUnicodeCharacter{211D}{\ensuremath{\mathbb{R}}}         %ℝ
\DeclareUnicodeCharacter{03BD}{\ensuremath{\nu}}                %ν
\DeclareUnicodeCharacter{03BE}{\ensuremath{\xi}}                %ξ
\DeclareUnicodeCharacter{03C0}{\ensuremath{\pi}}                %π
\DeclareUnicodeCharacter{03C1}{\ensuremath{\rho}}               %ρ
\DeclareUnicodeCharacter{03C3}{\ensuremath{\sigma}}             %σ
\DeclareUnicodeCharacter{03C4}{\ensuremath{\tau}}               %τ
\DeclareUnicodeCharacter{1D60}{\ensuremath{\varphi}}            %φ
\DeclareUnicodeCharacter{03C8}{\ensuremath{\psi}}               %ψ
\DeclareUnicodeCharacter{03C9}{\ensuremath{\omega}}             %ω
\DeclareUnicodeCharacter{03A9}{\ensuremath{\Omega}}             %Ω
\DeclareUnicodeCharacter{221A}{\ensuremath{\sqrt}}              %√
\DeclareUnicodeCharacter{2080}{\ensuremath{_0}}                 %₀
\DeclareUnicodeCharacter{2081}{\ensuremath{_1}}                 %₁
\DeclareUnicodeCharacter{2082}{\ensuremath{_2}}                 %₂
\DeclareUnicodeCharacter{2083}{\ensuremath{_3}}                 %₃
\DeclareUnicodeCharacter{2070}{\ensuremath{^0}}                 %⁰
\DeclareUnicodeCharacter{00B9}{\ensuremath{^1}}                 %ⁱ
\DeclareUnicodeCharacter{00B2}{\ensuremath{^2}}                 %²
\DeclareUnicodeCharacter{00B3}{\ensuremath{^3}}                 %³
\DeclareUnicodeCharacter{00B7}{\ensuremath{\cdot}}              %·
\DeclareUnicodeCharacter{2248}{\ensuremath{\approx}}            %≈
\DeclareUnicodeCharacter{2260}{\ensuremath{\neq}}               %≠
\DeclareUnicodeCharacter{211D}{\ensuremath{\mathbb{N}}}         %ℕ
\DeclareUnicodeCharacter{2102}{\ensuremath{\mathbb{C}}}         %ℂ
\DeclareUnicodeCharacter{2265}{\ensuremath{\ge}}                %≥
\DeclareUnicodeCharacter{2264}{\ensuremath{\le}}                %≤
\DeclareUnicodeCharacter{00F7}{\ensuremath{\pm}}                %÷
\DeclareUnicodeCharacter{2026}{\ensuremath{\dots}}              %…
